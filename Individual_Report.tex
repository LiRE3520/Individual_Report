\documentclass{article}
\usepackage{amsthm}
\usepackage{amsmath, amssymb}
\usepackage[margin=1in]{geometry}
\usepackage{ytableau}

\theoremstyle{definition}
\newtheorem{definition}{Definition}
\newtheorem{theorem}{Theorem}

\title{Partitions}
\author{Lewis Reed}
\date{March 2025}

\begin{document}

\maketitle

\section{What is a partition?}

\begin{definition}
    A \textbf{partition} $p(n)$ of a natural number $n$ is a way of writing $n$ as a sum of positive integer parts.
\end{definition}

\noindent
For example, the partitions of 4 look like this:

\[
\begin{aligned}
&4 \\
&3 + 1 \\
&2 + 2 \\
&2 + 1 + 1 \\
&1 + 1 + 1 + 1
\end{aligned}
\]

\noindent
Each number in a partition is called a part. The function $p(n)$ counts the total number of partitions of $n$.
There is no formula for $p(n)$, but Hardy and Ramanujan got very close with their asymptotic formula:

\[
p(n) \sim \frac{1}{4n\sqrt{3}}\exp\left(\pi\sqrt{\frac{2n}{3}}\right)
\]

\noindent
$p(n)$ can be given conditions to produce more specific results, where different results can be compared
and bijections can be drawn.

\begin{theorem}
    $p(n \mid \text{each part is odd}) = p(n \mid \text{each part is distinct})$
\end{theorem}
    
\begin{proof}
Starting with an odd-partition part, 'pair up' repeated parts and replace them with their double until there are
no repeated parts. Once all repeated parts are paired up, the parts of the partition are all distinct.
\newline For example, consider this partiton of 10 with only odd parts:

\[
5 + 3 + 1 + 1
\]

\noindent
Pair up the two 1s and replace them with 2 to get this new partition:

\[
5 + 3 + 2
\]

\noindent
There are no more repeated parts, so we are now left with a partiton of distinct parts.
\newline Starting with a distinct-part partition, 'split' any even parts and replace them with their two halves.

\[
5 + 3 + 2
\]

\noindent
Split up the 2 and replace it with two 1s to get this new partition:

\[
5 + 3 + 1 + 1
\]

\noindent
Every odd-part partition uniquely maps to a distinct-part partition and vice versa. This establishes a bijection.
\end{proof}

\newpage

\section{Representing partitions with Young diagrams}

\begin{definition}
    A \textbf{Young diagram} is a diagram used to represent a partition.
\end{definition}

\noindent
It is made up of squares, where each square represents
the value of 1. Each part is represented by a row of squares, with the rows going down in decreasing order.
\newline Here is the Young diagram for the partition of 10, 5 + 3 + 2:

\[
\begin{minipage}{0.3\textwidth}
    \ydiagram[]
        {5,3,2}
\end{minipage}
\begin{minipage}{0.3\textwidth}
    First row: 5 squares \\
    Second row: 3 squares \\
    Third row: 2 squares
\end{minipage}
\]

\noindent
These diagrams can be used to further analyse bijections between different types of partitions.

\begin{theorem}
    $p(n \mid \text{largest part has size $k$}) = p(n \mid \text{has $k$ parts})$
\end{theorem}

\begin{proof}
Consider the \textbf{conjugate} of the previous Young diagram, where the conjugate is the result of
flipping the diagram on its diagonal:

\[
\begin{minipage}{0.3\textwidth}
    \ydiagram[]
        {3,3,2,1,1}
\end{minipage}
\begin{minipage}{0.3\textwidth}
    This represents the partition 3 + 3 + 2 + 1 + 1 which is also a partiton of 10.
\end{minipage}
\]

\noindent
When flipping the diagram, the first row, which is the largest part, becomes the first column, which is the
number of parts. In the previous example, the largest part of the original partition has size 5, and the number of
parts of the conjugate is 5.
\newline Every Young diagram uniquely maps to its conjugate and vice versa. This establishes a bijection.
\end{proof}

\section{Representing partitions with generating functions}

\begin{definition}
    A \textbf{generating function} is a way to represent a sequence of numbers as a power series. 
\end{definition}

\noindent
Since the geometric series identity gives:

\[
\frac{1}{1 - x} = 1 + x + x^2 + x^3 + x^4 + \dots
\]

\noindent
and similarly,

\[
\frac{1}{1 - x^2} = 1 + x^2 + x^4 + x^6 + x^8 + \dots,
\]

\noindent
each term represents including that power of \( x \) any number of times. Extending this pattern, we obtain
the partition generating function:

\[
\prod_{n=1}^{\infty} \frac{1}{1 - x^n}
\]

\noindent
where each factor \( \frac{1}{1 - x^n} \) accounts for the possibility of including the number \( n \) any
number of times in a partition.

\newpage

To find $p(6)$, the number of partitions of 6, we expand the first few terms of the generating function:
    
\begin{align}
\prod_{n=1}^{\infty} \frac{1}{1 - x^n} &= \frac{1}{1-x} \cdot \frac{1}{1-x^2} \cdot \frac{1}{1-x^3} \cdot \frac{1}{1-x^4} \cdot \frac{1}{1-x^5} \cdot \frac{1}{1-x^6} \cdot \ldots \\
\end{align}

Each factor can be expanded as follows:
\begin{align}
\frac{1}{1-x} &= 1 + x + x^2 + x^3 + x^4 + x^5 + x^6 + \ldots \\
\frac{1}{1-x^2} &= 1 + x^2 + x^4 + x^6 + \ldots \\
\frac{1}{1-x^3} &= 1 + x^3 + x^6 + \ldots \\
\frac{1}{1-x^4} &= 1 + x^4 + \ldots \\
\frac{1}{1-x^5} &= 1 + x^5 + \ldots \\
\frac{1}{1-x^6} &= 1 + x^6 + \ldots
\end{align}

When we multiply these expansions, the coefficient of $x^6$ in the product will correspond to $p(6)$. To find this coefficient, we need to identify all combinations of terms whose powers sum to 6:

\[
\begin{align}
\item $x^6$ from $\frac{1}{1-x}$: represents the partition $6 = 1+1+1+1+1+1$
\item $x^4 \cdot x^2$ from $\frac{1}{1-x}$ and $\frac{1}{1-x^2}$: represents $6 = 1+1+1+1+2$
\item $x^2 \cdot x^4$ from $\frac{1}{1-x}$ and $\frac{1}{1-x^4}$: represents $6 = 1+1+4$
\item $x^2 \cdot x^2 \cdot x^2$ from $\frac{1}{1-x^2}$: represents $6 = 2+2+2$
\item $x^3 \cdot x^3$ from $\frac{1}{1-x^3}$: represents $6 = 3+3$
\item $x^3 \cdot x^2 \cdot x^1$ from $\frac{1}{1-x^3}$, $\frac{1}{1-x^2}$, and $\frac{1}{1-x}$: represents $6 = 3+2+1$
\item $x^3 \cdot x^1 \cdot x^1 \cdot x^1$ from $\frac{1}{1-x^3}$ and $\frac{1}{1-x}$: represents $6 = 3+1+1+1$
\item $x^2 \cdot x^2 \cdot x^1 \cdot x^1$ from $\frac{1}{1-x^2}$ and $\frac{1}{1-x}$: represents $6 = 2+2+1+1$
\item $x^4 \cdot x^1 \cdot x^1$ from $\frac{1}{1-x^4}$ and $\frac{1}{1-x}$: represents $6 = 4+1+1$
\item $x^5 \cdot x^1$ from $\frac{1}{1-x^5}$ and $\frac{1}{1-x}$: represents $6 = 5+1$
\item $x^6$ from $\frac{1}{1-x^6}$: represents the partition $6 = 6$
\end{align}
\]

Counting these combinations, we find that $p(6) = 11$. The complete list of partitions of 6 is:
\begin{align}
6 &= 6\\
6 &= 5+1\\
6 &= 4+2\\
6 &= 4+1+1\\
6 &= 3+3\\
6 &= 3+2+1\\
6 &= 3+1+1+1\\
6 &= 2+2+2\\
6 &= 2+2+1+1\\
6 &= 2+1+1+1+1\\
6 &= 1+1+1+1+1+1
\end{align}

Note that we've actually missed two partitions in our enumeration above: $6 = 3+1+1+1$ and $6 = 2+2+1+1$. This is because we need to consider all possible ways to obtain $x^6$ when multiplying the infinite series together.

More rigorously, we can compute the first few terms of the product:
\begin{align}
(1 + x + x^2 + \ldots)(1 + x^2 + x^4 + \ldots)(1 + x^3 + x^6 + \ldots)\ldots
\end{align}

This gives us:
\begin{align}
1 + x + 2x^2 + 3x^3 + 5x^4 + 7x^5 + 11x^6 + \ldots
\end{align}

Thus, the coefficient of $x^6$ is 11, confirming that $p(6) = 11$.

\end{document}