\documentclass{article}
\usepackage{amsthm}
\usepackage{amsmath, amssymb}
\usepackage[margin=1in]{geometry}

\theoremstyle{definition}
\newtheorem{definition}{Definition}
\newtheorem{theorem}{Theorem}

\title{Partitions}
\author{Lewis Reed}
\date{March 2025}

\begin{document}

\maketitle

\section{What is a partition?}

\begin{definition}
    A \textbf{partition} $p(n)$ of a natural number $n$ is a way of writing $n$ as a sum of positive integer parts.
\end{definition}

\noindent
For example, the partitions of 4 look like this:

\[
\begin{aligned}
&4 \\
&3 + 1 \\
&2 + 2 \\
&2 + 1 + 1 \\
&1 + 1 + 1 + 1
\end{aligned}
\]

\noindent
Each number in a partition is called a part. The function $p(n)$ counts the total number of partitions of $n$.
There is no formula for $p(n)$, but Hardy and Ramanujan got very close with their asymptotic formula:

\[
p(n) \sim \frac{1}{4n\sqrt{3}}\exp\left(\pi\sqrt{\frac{2n}{3}}\right)
\]

\noindent
$p(n)$ can be given conditions to produce more specific results, where different results can be compared
and bijections can be drawn.

\begin{theorem}
    $p(n \mid \text{each part is odd}) = p(n \mid \text{each part is distinct})$
\end{theorem}
    
\begin{proof}
Starting with an odd-partition part, 'pair up' repeated parts and replace them with their double until there are
no repeated parts. Once all repeated parts are paired up, the parts of the partition are all distinct.
\newline For example, consider this partiton of 10 with only odd parts:

\[
5 + 3 + 1 + 1
\]

\noindent
Pair up the two 1s and replace them with 2 to get this new partition:

\[
5 + 3 + 2
\]

\noindent
There are no more repeated parts, so we are now left with a partiton of distinct parts.
\newline Starting with a distinct-part partition, 'split' any even parts and replace them with their two halves.

\[
5 + 3 + 2
\]

\noindent
Split up the 2 and replace it with two 1s to get this new partition:

\[
5 + 3 + 1 + 1
\]

\noindent
Every odd-part partition uniquely maps to a distinct-part partition and vice versa. This establishes a bijection.
\end{proof}

\newpage

\section{Representing partitions with Young diagrams}


\end{document}